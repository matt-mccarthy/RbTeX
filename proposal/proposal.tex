\documentclass{article}

\usepackage{amsmath}
\usepackage{listings}

\newcommand{\inlinecode}[1]{\texttt{#1}}
\newcommand{\luatex}{\inlinecode{luatex}\ }
\newcommand{\findent}{\leavevmode{\parindent=1.3em\indent}}

\def\RbTeX{{\rm\kern-.125emR\!{\scriptsize\lower-.5ex\hbox{B}}\!T\kern-.1667em\lower.5ex\hbox{E}\kern-.125emX}}

\topmargin=-1in
\textheight=9.2in
\textwidth=168mm
\oddsidemargin=-0.2in
\evensidemargin=-0.2in

\pagenumbering{gobble}
\title{CPSC 498 Proposal}
\author{Steven Rosendahl}
\date{}
\begin{document}
\maketitle
\section{Abstract}

Modern \LaTeX\ distributions include a tool called \luatex that allows users to dynamically
produce content via use of Lua code. Unfortunately, the Lua standard libraries do not have as much
functionality as other popular scripting languages, such as Ruby. The goal of this project is to
incorporate Ruby into \LaTeX\ in a manner similar to \luatex, but with the power and
simplicity of Ruby over Lua.

\section{Specification}

\findent The current \luatex specification allows users to use several environments for writing and
running Lua scripts. In addition, \luatex provides a built in library called \inlinecode{tex} that allows
output to be printed straight to the \LaTeX\ document. The library, called \RbTeX, will provide similar
functionality through a gem called \inlinecode{rbtex}. In addition, the entire Ruby standard library will
be available for use; \RbTeX\ documents that need to interact directly with the system will most likely
need to be compiled using the \inlinecode{--shell-escape} flag.\\

To use the library, users will need to have a Ruby version in the path. The code will be pre-processed,
and inserted directly into the \TeX\ code before \inlinecode{pdflatex} is called on the document. In
addition, users will be provided with several ways in which to interact with Ruby from the \TeX\
environment:

\begin{enumerate}
\item \inlinecode{inrbtex\{\}}: This command will provide a way for a use to execute one line of Ruby code
at a time, or call a predefined function.
\item \inlinecode{rbtex\{\}}: This command will provide a way to write multiple lines of Ruby code. Any
functions defined in this section will be globally defined, so they can be called in the
\inlinecode{inrbtex\{\}} environment and in other \inlinecode{rbtex\{\}} environments.
\end{enumerate}

The library will come with a program called \inlinecode{rbtex} that complies the provided \LaTeX\
document, much like the \inlinecode{luatex} command.

\end{document}
