\documentclass{article}
\usepackage{rubylatex}

\begin{document}

\begin{rbtex}

def printGrahm num
    tstr = "#{num}^{"

    i = 0
    while i < 64
        tstr << "#{num}^{"
        i = i + 1
    end
    tstr << "#{num}"
    while i >= 0
        tstr << "}"
        i = i - 1
    end
    Tex.print Tex.cmath(tstr)
end

\end{rbtex}


This is native \LaTeX\ !

\begin{rbtex}
module TM

    def TM.hw
        Tex.print ('hello, world!')
    end

    class TestClass

        def initialize
            Tex.print 'class initialized!'
        end

        def doThing(arg0)
            Tex.print 'doing thing...'
            Tex.print "Here's arg0 #{arg0}"
        end

    end

end

TM.hw
\end{rbtex}

We are about to use a class inside a module. This should be fun!\\

\begin{rbtex}
puts 'hello, world!'
mtm = TM::TestClass.new
mtm.doThing "afhalkdjfhaslkdj"
\end{rbtex}

\end{document}
