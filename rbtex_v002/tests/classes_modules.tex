\documentclass{article}
\usepackage{rubylatex}

\begin{document}
\RbTeX\ supports classes and modules, which makes it ideal for all you object oriented needs.\\

\begin{rbtex}
module Shape

    class Square
        def initialize(width, height)
            @width = width
            @height = height
        end

        def area
            @width * @height
        end
    end

end
\end{rbtex}

We've already defined some of the Ruby stuff we want to later use. Let's go ahead and use it!

\begin{rbtex}
sqr = Shape::Square.new(10, 44)
Tex.print "My Square's area is #{sqr.area}!"
\end{rbtex}

\end{document}
